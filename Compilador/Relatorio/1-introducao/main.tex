\mychapter{Projeto Cimplex}
\label{Cap:Projeto}

O projeto consiste em definir uma linguagem e a partir dessa contruir o compilador em C. O nome da linguagem é Cimplex.

\section{A linguagem}
\label{sec:liguagem_informal}

A linguagem base do compilador está descrita abaixo, em uma primeira instância, através de uma definição informal:

\begin{itemize}
\item{tipagem}: números naturais (int), números de ponto flutuante (float), strings (string), booleano (bool), vetor (array[1]) e matriz (variavel[1][1])
\item{estrutura inicial de programa}: 
	\begin{lstlisting}
		begin do
			-Programa-
		end
	\end{lstlisting}
   
\item{Operações Aritméticas}:
	Além das quatro básicas, também é possível fazer potências:
	\begin{lstlisting}
		int var1 = 2;
		int var2 = 4;
		int var3 = var2 ^ var1;
	\end{lstlisting}

\item{Declarações}: 
	\begin{lstlisting}
		int var1 = 0;
	\end{lstlisting}
    Sendo possível a declaração múltipla:
    \begin{lstlisting}
		int var1 = 1, var2, var3 = 3;
		float float1 = 0.1, float2 = 1;
	\end{lstlisting}
    No caso de arrays e matrizes, é possível inicia-los com literais:
    \begin{lstlisting}
		string array[20] = ["olha", "esse", "teste", "!"];
		bool matrix[3][3] = [
								[false, true, true],
								[true, false, true],
								[true, true, false]
							];
	\end{lstlisting}
\item{Comparações}: 
	\begin{lstlisting}
		var1 <= var2;
	\end{lstlisting}
    Se os valores comparados forem numeros, é possível compará-los com:
    \begin{lstlisting}
		<=, >=, <, >, ==, !=
	\end{lstlisting}
    Se eles forem booleanos, só é possível compará-los com:
    \begin{lstlisting}
		==, !=
	\end{lstlisting}
    
\item{Condicionais}: 
	\begin{lstlisting}
		if(var1 == 0) do
        	-Programa-
		endif
	\end{lstlisting}
    
\item{Iteradores}: 
	\begin{lstlisting}
		while(true) do
        	i = i + 1;
		endwhile
	\end{lstlisting}

\item{Funções}: 
	Declaração:
	\begin{lstlisting}
		function int func1(float a, float b) do
        	return 1;
		endfunction
	\end{lstlisting}
    
    Chamada:
    \begin{lstlisting}
		int var_int = func1(1.0, 2.0);
	\end{lstlisting}
    
\item{Entrada}: 
	\begin{lstlisting}
		scan(var1, var2, var3[0]);
	\end{lstlisting}
    
\item{Saída}: 
	\begin{lstlisting}
		print("var1=", var1, "var2=", var2, "var3[0]=", var3[0]);
	\end{lstlisting}
    
\item{Seleção}: 
	\begin{lstlisting}
		when(var1)
			is(1) do
				print("teste1");
			break;
			is(2) do
				print("teste2");
			break;
			default do
				print("erro");
			break;
		endwhen
	\end{lstlisting}
    
\item{Comentários}: 
	\begin{lstlisting}
		##
      	Sou um comentario de bloco!
    ##
        
    #Sou um comentario de linha!
	\end{lstlisting}
    
\item{Estruturas heterogêneas}: 
	\begin{lstlisting}
		struct Caixa do
			int inteiro;
			string nome;
		endstruct
	\end{lstlisting}
    Para acessar uma variável da caixa:
    \begin{lstlisting}
    	Caixa->inteiro = 20;
		int var1 = Caixa->inteiro;
	\end{lstlisting}
    Essa struct servirá apenas como uma variável simples. Se o usuário quiser outra caixa, ele deverá criar outra caixa:
    \begin{lstlisting}
		struct Caixa2 do
			int inteiro;
			string nome;
		endstruct
	\end{lstlisting}
\end{itemize}
